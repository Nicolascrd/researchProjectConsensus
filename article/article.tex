\documentclass[11pt, twocolumn]{article}
\usepackage[T1]{fontenc}
\usepackage{amsmath, amssymb}
\usepackage{graphicx}
\usepackage[utf8]{inputenc}
\usepackage{fdsymbol}
\usepackage{textgreek}
\usepackage{natbib}
\usepackage{hyperref}
\usepackage{url}
\usepackage{array}
\usepackage{csquotes}
\usepackage{caption}
\hypersetup{
    colorlinks,
    citecolor=black,
    filecolor=black,
    linkcolor=black,
    urlcolor=black
}

\title{Decentralized state machine using Nakamoto (probabilistic) consensus\\\medskip Research Project 2022}
\author{Nicolas COURAUD (nicolas.couraud@etu.emse.fr)\\École Nationale Supérieure des Mines de Saint-Étienne}

\begin{document}

\maketitle
\onecolumn
\section*{Abstract}

As defined by Schneider \cite{stateMachine}, the state machine approach is a general method for implementing a fault-tolerant service
by replicating servers and coordinating client interactions with server replicas.

In this article, we look at state machines as a way to hold in multiple servers copies of the same record. That record, in our implementation, is a standard 
key-value data structure (\href{https://go.dev/blog/maps}{a Go map}).

We show how one can build a decentralized state machine using Nakamoto Consensus (in our implementation we used the Snowball consensus algorithm), and we study the advantages 
and drawbacks of such an approach, compared to classical consensus algorithms like Raft \cite{understandable}.

The implementation of the project is available at \href{https://github.com/Nicolascrd/distributed-state-machine}{github.com/Nicolascrd/distributed-state-machine}


\tableofcontents
\section{Traditional state machine replication}

\subsection{Introduction}

Introduction
Introduction
Introduction
Introduction
Introduction
Introduction
Introduction

\subsection{Byzantine fault-tolerance with classical consensus}

Byzantine
Byzantine
Byzantine
Byzantine
Byzantine
Byzantine
Byzantine
Byzantine

\section{Nakamoto (probabilistic) state machine replication}
\subsection{The Nakamoto Consensus}

The Naka consensus
The Naka consensus
The Naka consensus
The Naka consensus
The Naka consensus
The Naka consensus
The Naka consensus

\subsection{My Nakamoto state machine implementation}

Nakamoto state machine
Nakamoto state machine
Nakamoto state machine
Nakamoto state machine
Nakamoto state machine
Nakamoto state machine
Nakamoto state machine
Nakamoto state machine
Nakamoto state machine
Nakamoto state machine
Nakamoto state machine
Nakamoto state machine
Nakamoto state machine

\section{Performance Evaluation}
\subsection{Probability of not reaching consensus in Nakamoto Consensus}

Markov
Markov
Markov
Markov
Markov
Markov
Markov
Markov
Markov
Markov

\subsection{Number of requests required in regular functioning}

Number of requests
Number of requests
Number of requests
Number of requests
Number of requests
Number of requests
Number of requests
Number of requests
Number of requests
Number of requests
Number of requests
Number of requests

\subsection{In case of a failure}

Failure ?
Failure ?
Failure ?
Failure ?
Failure ?
Failure ?
Failure ?
Failure ?
Failure ?
Failure ?

\section{Conclusion}

Conclusion
Conclusion
Conclusion
Conclusion
Conclusion
Conclusion
Conclusion
Conclusion
Conclusion

\bibliographystyle{alpha}
\bibliography{biblio}

\end{document}
