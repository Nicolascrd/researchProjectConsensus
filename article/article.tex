\documentclass[11pt, twocolumn]{article}
\usepackage[T1]{fontenc}
\usepackage{amsmath, amssymb}
\usepackage{graphicx}
\usepackage[utf8]{inputenc}
\usepackage{fdsymbol}
\usepackage{textgreek}
\usepackage{natbib}
\usepackage{hyperref}
\usepackage{url}
\usepackage{array}
\usepackage{csquotes}
\usepackage{caption}
\hypersetup{
    colorlinks,
    citecolor=black,
    filecolor=black,
    linkcolor=black,
    urlcolor=black
}

\title{Decentralized state machine using Nakamoto (probabilistic) consensus\\\medskip Research Project 2022}
\author{Nicolas COURAUD (nicolas.couraud@etu.emse.fr)\\École Nationale Supérieure des Mines de Saint-Étienne}

\begin{document}

\maketitle
\onecolumn
\section*{Abstract}

As defined by Schneider \cite{stateMachine}, the state machine approach is a general method for implementing a fault-tolerant service
by replicating servers and coordinating client interactions with server replicas.

In this article, we look at state machines as a way to hold in multiple servers copies of the same record. That record, in my implementation, is a standard 
key-value data structure (\href{https://go.dev/blog/maps}{a Go map}).

I show how one can build a decentralized state machine using Nakamoto Consensus (in our implementation we used the Snowball consensus algorithm), and I study the advantages 
and drawbacks of such an approach, compared to classical consensus algorithms like Raft \cite{understandable}.

The implementation of the project is available at \href{https://github.com/Nicolascrd/distributed-state-machine}{github.com/Nicolascrd/distributed-state-machine}


\tableofcontents
\section{Traditional state machine replication}

\subsection{Introduction}

Decentralized state machine systems are currently built, for the most part, around classical consensus algorithms.

In classical consensus, all of the nodes have to know each other. It is a permissioned model, which means that nodes have to get an approval to get into the network. 
That is very normal for private infrastructure but might not work for public infrastructure. Because all nodes must communicate with each other, the communication cost in generally considered high.

Many classical consensus algorithms exists, the most famous one being Paxos \cite{parliament}. In my project, I used the Raft consensus algorithm \cite{understandable} because it has the same performances
as Paxos, and is simpler to understand and implement. 

In Raft, each node can be either Leader, Follower or Candidate. In regular functioning, one node will be the leader and all the others will follow. 
The Leader node sends regular heartbeat requests to notify followers that it is still running. If they stop receiving the heartbeat, they switch to the follower status.

The client can ask any node to add a log to the record. 
The request includes the number of the log that the client wants to add and the log himself. If the client as a node which is not the leader, the request is transfered to the leader. If there is already a record at that number,
the leader returns an error message to the client. Otherwise, the leader updates its record and asks all his followers to update themselves as well. 

When requesting a log, the client can also query any node in the network. The node just responds with the log at that number in the local record that it is holding.

The only parameter than I included in my implementation is \emph{updateSystem}, which is a boolean. If True, if the leader sends heartbeats and a node does not reply, it will be eliminated from the network in the knowledge of all the nodes. This make it possible to 
crash a majority of nodes and still have the state machine running and usable on all the surviving nodes.

\subsection{Byzantine fault-tolerance (BFT) with classical consensus}

Contrary to Crash Fault, Byzantine Fault \cite{byzantine} is a type of failure where you consider that the node can fail in any possible way. In practice, it means that the node will stay up and can send malicious requests to 
interfere with the functioning of the network. 

My implementation of the state machine does not tolerate Byzantine Faults at all. With only one byzantine node, it already endangers the whole system. Indeed, the malicious node can "elect himself" in Raft, and then send heartbeats to all 
the node in the network, including the legitimate leader to become the leader de facto. Then, it can ignore requests, reply whatever it wants and write any log it wants in any node it wants.

To tackle this issue, BFT algorithms were designed \cite{pbft}. They include a layer of authentification, but are not necessarily much slower. However, there is a maximum number of malicious nodes that the algorithm can handle safely \cite{partialSynchrony}. 
For example, assuming partial synchrony of the nodes, consensus protocols can handle t crash failures with 2t+1 nodes but t byzantine failures with no less than 3t+1 nodes.   

\section{Nakamoto (probabilistic) state machine replication}
\subsection{The Nakamoto Consensus}

The Naka consensus
The Naka consensus
The Naka consensus
The Naka consensus
The Naka consensus
The Naka consensus
The Naka consensus

\subsection{My Nakamoto state machine implementation}

Nakamoto state machine
Nakamoto state machine
Nakamoto state machine
Nakamoto state machine
Nakamoto state machine
Nakamoto state machine
Nakamoto state machine
Nakamoto state machine
Nakamoto state machine
Nakamoto state machine
Nakamoto state machine
Nakamoto state machine
Nakamoto state machine

\section{Performance Evaluation}
\subsection{Probability of not reaching consensus in Nakamoto Consensus}

Markov
Markov
Markov
Markov
Markov
Markov
Markov
Markov
Markov
Markov

\subsection{Number of requests required in regular functioning}

Number of requests
Number of requests
Number of requests
Number of requests
Number of requests
Number of requests
Number of requests
Number of requests
Number of requests
Number of requests
Number of requests
Number of requests

\subsection{In case of a failure}

Failure ?
Failure ?
Failure ?
Failure ?
Failure ?
Failure ?
Failure ?
Failure ?
Failure ?
Failure ?

\section{Conclusion}

Conclusion
Conclusion
Conclusion
Conclusion
Conclusion
Conclusion
Conclusion
Conclusion
Conclusion

\bibliographystyle{alpha}
\bibliography{biblio}

\end{document}
